\documentclass{report}

\input{preamble}
\input{macros}
\input{letterfonts}

\title{\Huge{Magnet Precalculus C}\\Semester Exam Review}
\author{\huge{Devin D. Droddy}}
\date{}

\begin{document}

\maketitle
\newpage% or \cleardoublepage
% \pdfbookmark[<level>]{<title>}{<dest>}
\pdfbookmark[section]{\contentsname}{toc}
\tableofcontents
\pagebreak

\chapter{}
\section{Solving Polynomials}
\qs{1a}{Solve for \(x\) where \(2x^3=-3x^2+2x\).}
\sol Subtract \(-3x^2+2x\) from both sides to find that \(2x^3+3x^2-2x=0\). Factor \(x\) from that identity to find that \(x(2x^2+3x-2)=0\). Multiplying 2 by -2 tells us that we need to find two numbers which sum to 3 and produce -4. These numbers are -1 and 4. \(2x+4\) can be simplified to \(x+2\). We now know that the factors of \(2x^2+3x-2\) are \(x+2\) and \(2x-1\). Therefore, \(x(2x^2+3x-2)=x(x+2)(2x-1)\). The values of \(x\) are 0, -2, and \(\frac{1}{2}\).
\qs{1b}{Solve for \(x\) where \(x^2=3x-1\).}
\sol Subtract \(3x-1\) from both sides to find that \(x^2-3x+1=0\). There are no two numbers which sum to -3 and produce 1, so we must use the quadratic formula. \(x=\frac{3\pm\sqrt{9-4(1*1)}}{2}=\underset{Solution}{\boxed{\frac{3\pm\sqrt{5}}{2}}}\).
\section{Domain and Range of Functions}
\qs{2a}{Find the domain and range of the function \(f(x)=x^2+sqrt{x-3}\).}
\sol The more restrictive function is the square root function, so we must look there to find our domain restriction. We can see that the square root is translated 3 units to the right, so the domain is \([3,\infty)\). At \(x=3\), \(y=3^2+sqrt{3-3}=3^2=9\), so the point at which the curve ends is \((3, 9)\). The range is \([9,\infty)\)
\qs{2b}{Find the domain and range of the function \(f(x)=\frac{x-5}{x^2-x-20}\).}
\sol \(x^2-x-20\) can be factored into \((x+4)(x-5)\), and since \(x-5\) is in the numerator and denominator, it can be removed. We are left with \(f(x)=\frac{1}{x+4}\). This is a reciprocal function, translated 4 units left. This means that its domain is \(\mathbb{R};x\neq -4\), and its range is \(\mathbb{R};y\neq 0\), since its asymptotes are at \(y=0\) and \(x=-4\).
\section{Increasing, Decreasing, and Constant Intervals}
\qs{3a}{Determine the intervals over which the function \(f(x)=(x^2-4)^2\) is increasing, decreasing, or constant.}
\sol \((x^2-4)^2\) can be factored into \((x^2-4)(x^2-4)\), which can be further factored into \((x+2)(x-2)(x+2)(x-2)=(x+2)^2(x-2)^2\). The degree of the function is 4, which is positive and even, so the end behavior of the function is \(\underset{x\to\infty}{\lim}f(x)=\infty;\underset{x\to -\infty}{\lim}f(x)=\infty\). The zeros -2 and 2 both have a multiplicity of 2, so they serve as relative minimums/maximums. \(f(x)\) decreases from \(-\infty\), so \((-2,0)\) is a relative minimum. \(f(x)\) increases to \(\infty\), so \((2,0)\) must also be a relative minimum. This means there must be a relative maximum somewhere in the middle between these two points. \(2-(-2)=4;\frac{4}{2}=2;-2+2=0\), therefore the x-value of the relative maximum must be 0. \(f(0)=(0^2-4)^2=(-4)^2=16\), therefore the relative maximum is located at \((0, 16)\). With our 2 relative minimums and relative maximum located, we can find the increasing and decreasing intervals. Increasing: \((-2, 0)u(2,\infty)\). Decreasing: \((-\infty,-2)u(0,2)\).
\qs{3b}{Determine the intervals over which the function \(f(x)=|x-1|+1\) is increasing, decreasing, or constant.}
\sol This absolute value function is translated 1 unit up and 1 unit to the right. Therefore, its increasing interval is \((1,\infty)\), and its decreasing interval is \((-\infty,1)\)
\qs{3c}{Determine the intervals over which the function \(g(x)=\frac{x^2}{x^2-4}\) is increasing, decreasing, or constant.}
\sol In this reciprocal function, the degree of the numerator is equal to that of the denominator, so the horizontal asymptote is at \(y=1\), since the quotient of the two leading coefficients is \(\frac{1}{1}\). There are two vertical asymptotes, since the denominator has a degree of 2. \(x^2-4\) can be factored into \((x+2)(x-2)\), making the zeros of the expression 2 and -2. Therefore, the vertical asymptotes are at \(x=2\) and \(x=-2\). We can determine direction by calculating a few values. \(g(-3)=\frac{(-3)^2}{(-3)^2-4}=\frac{9}{9-4}=\frac{9}{5}>1\), so on the interval \((-\infty,-2)\), \(g(x)\) increases. \(g(-1)=\frac{(-1)^2}{(-1)^2-4}=\frac{1}{1-4}=\frac{1}{-3}<1\), so on the interval \((-2,0)\), \(g(x)\) increases. In order to follow the vertical asymptote \(x=2\), \(g(x)\) must decrease on the interval \((0,2)\). \(g(3)=\frac{3^2}{3^2-4}=\frac{9}{9-4}=\frac{9}{5}>1\), therefore on the interval \((2,\infty)\), \(g(x)\) decreases.
\qs{3d}{Determine the intervals over which the function \(h(x)=\sqrt[3]{x+1}\) is increasing, decreasing, or constant.}
\sol This third root function is translated 1 unit to the left. However, it isn't reflected, so it is always increasing.
\section{Even and Odd Functions}
\qs{4a}{Determine whether the function \(f(x)=x^5+3x-4\) is even, odd, or neither}
\end{document}
