\documentclass{report}

\input{preamble}
\input{macros}
\input{letterfonts}

\title{\Huge{Magnet Precalculus C}\\Semester Exam Review}
\author{\huge{Devin D. Droddy}}
\date{}

\begin{document}

\maketitle
\newpage% or \cleardoublepage
% \pdfbookmark[<level>]{<title>}{<dest>}
\pdfbookmark[section]{\contentsname}{toc}
\tableofcontents
\pagebreak

\chapter{}
\section{Solving polynomials}
\qs{}{Solve for \(x\) where \(2x^3=-3x^2+2x\)}
\sol Subtract \(-3x^2+2x\) from both sides to find that \(2x^3+3x^2-2x=0\). Factor \(x\) from that identity to find that \(x(2x^2+3x-2)=0\). Multiplying 2 by -2 tells us that we need to find two numbers which sum to 3 and produce -4. These numbers are -1 and 4. \(2x+4\) can be simplified to \(x+2\). We now know that the factors of \(2x^2+3x-2\) are \(x+2\) and \(2x-1\). Therefore, \(x(2x^2+3x-2)=x(x+2)(2x-1)\). The values of \(x\) are 0, -2, and \(\frac{1}{2}\).
\qs{}{Solve for \(x\) where \(x^2=3x-1\)}
\sol Subtract \(3x-1\) from both sides to find that \(x^2-3x+1=0\). There are no two numbers which sum to -3 and produce 1, so we must use the quadratic formula. \(x=\frac{3\pm\sqrt{9-4(1*1)}}{2}=\underset{Solution}{\boxed{\frac{3\pm\sqrt{5}}{2}}}\).
\section{Domain and range of functions}
\qs{}{Find the domain and range of the function \(f(x)=x^2+sqrt{x-3}\)}
\sol The more restrictive function is the square root function, so we must look there to find our domain restriction. We can see that the square root is translated 3 units to the right, so the domain is \([3,\infty)\). At \(x=3\), \(y=3^2+sqrt{3-3}=3^2=9\), so the point at which the curve ends is \((3, 9)\). The range is \([9,\infty)\)
\qs{}{Find the domain and range of the function \(f(x)=\frac{x-5}{x^2-x-20}\)}
\sol \(x^2-x-20\) can be factored into \((x+4)(x-5)\), and since \(x-5\) is in the numerator and denominator, it can be removed. We are left with \(f(x)=\frac{1}{x+4}\). This is a reciprocal function, translated 4 units left. This means that its domain is \(\mathbb{R};x\neq -4\), and its range is \(\mathbb{R};y\neq 0\), since its asymptotes are at \(y=0\) and \(x=-4\).
\section{Increasing, decreasing, and constant intervals}
\qs{}{Determine the intervals over which the function \(f(x)=(x^2-4)^2\) is increasing, decreasing, or constant}
\end{document}