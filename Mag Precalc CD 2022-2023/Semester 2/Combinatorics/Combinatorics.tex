\documentclass{report}

\input{preamble}
\input{macros}
\input{letterfonts}

\title{\Huge{Magnet Precalculus D\\Combinatorics}}
\author{\huge{Devin D. Droddy}}
\date{}

\begin{document}

\maketitle
\newpage% or \cleardoublepage
% \pdfbookmark[<level>]{<title>}{<dest>}
\pdfbookmark[section]{\contentsname}{toc}
\tableofcontents
\pagebreak

\chapter{Intro to Combinatorics}

\section{Counting}

If decision $M$ can be made in $x$ number of ways and decision $N$ can be made in $y$ number of ways, then the number of ways to make both decisions is $x \cdot y$.

\ex{Counting}{
	The ice cream shop offers a choice of a 3 cone sizes, 15 flavors, and 8 toppings. How many cones are possible if you can only choose one flavor and one topping?

	\[3 \cdot 15 \cdot 8 = 360\]

	There are 360 cones possible.
}

\section{Factorials}

\dfn{Factorial}{
	A factorial of some number $n$, denoted as $n!$, is the product of all natural numbers from 1 to $n$. For example, $5! = 5 \cdot 4 \cdot 3 \cdot 2 \cdot 1 = 120$.
}

\section{Permutations}

\dfn{Permutation}{
	A permutation of objects, denoted as $_nP_r$, is an arrangement of $r$ objects chosen from a set of $n$ objects. The number of possible permutations of size $r$ is denoted as $_nP_r = \frac{n!}{(n-r)!}$.
}

\nt{$0! = 1$. This means that if $n=r$, then $_nP_r=n!$, since $n-r=0$, so $\frac{n!}{(n-r)!}=\frac{n!}{0!}=n!$.}

\ex{Permutation}{
	There are 16 players on the baseball team. How many ways can the coach make a 9-player batting order?

	\[
		n=16 \\
		r=9\\
		_nP_r=_{16}P_9=\frac{16!}{(16-9)!}=\frac{16!}{7!}=\frac{16!}{5040}=\frac{20922789888000}{5040}=4151347200\]\\
	There are 4,151,347,200 ways to make a 9-player batting order.
}

\section{Combinations}

\dfn{Combination}{
	A combination of objects, denoted as $_nC_r$, is a selection of $r$ objects chosen from a set of $n$ objects. The number of possible combinations of size $r$ is denoted as $_nC_r = \frac{n!}{r!(n-r)!}$. \textbf{This is different from a permutation because you can have multiple of the same object.}
}

\chapter{Theoretical Probability}

\section{Intro to Probability}

Probability is the measure of how likely an event is to occur. The set of all possible outcomes is called the \textbf{sample space}. For equally likely outcomes, the probability of an event $E$ is given by the formula $P(E)=\frac{\text{number of outcomes in }E}{\text{number of outcomes in the sample space}}$.

\ex{Simple Event}{
	A jar contains 32 red marbles and 28 blue marbles. The probability that a randomly selected marble is red is $\frac{32}{32+28}=\frac{32}{60}=\frac{8}{15}$.
}

\section{Compound Events}

A compound event is an event that consists of two or more simple events. There are two kinds of compound events: \textbf{independent} and \textbf{dependent}. A compound event is independent when the outcome of one event does not affect the outcome of the other event. A compound event is dependent when the outcome of one event does affect the outcome of the other event. In both cases, the probability of a compound event is the product of the probabilities of the simple events.

\ex{Independent Event}{
	A jar contains 32 red marbles and 28 blue marbles. If you select a marble, replace it, and then select another marble, the probability that both marbles are red is $\frac{32}{60} \cdot \frac{32}{60} = \frac{1024}{3600} = \frac{256}{900}$.
}

\ex{Dependent Event}{
	A jar contains 32 red marbles and 28 blue marbles. If you select a marble, do not replace it, and then select another marble, the probability that both marbles are red is $\frac{32}{60} \cdot \frac{31}{59} = \frac{992}{3540} = \frac{248}{885}$.
}

\nt{In the dependent event example, the effect of the first event is that the sample space is reduced by one.}

\chapter{Venn Diagrams}

This is a venn diagram:

\def\circleA{(180:1.75cm) circle (2.5cm)}
\def\circleB{(0:1.75cm) circle (2.5cm)}
\def\rectangleU{(5,-3) rectangle (-5,3)}
\def\circleAthree{(150:1.75cm) circle (2.5cm)}
\def\circleBthree{(30:1.75cm) circle (2.5cm)}
\def\circleCthree{(270:1.75cm) circle (2.5cm)}
\def\rectangleUthree{(5,-5) rectangle (-5,5)}
\begin{tikzpicture}
	\draw \rectangleU;
	\draw \circleA node {$A$};
	\draw \circleB node {$B$};
\end{tikzpicture}

You can specify many different areas within the diagram, such as only where $A$ and $B$ overlap, or everywhere other than $A$. There is standard mathematical notation to represent these areas.

\section{Basic Venn Diagram Notation}

First, $\cup$, the \textbf{or} operator. This is the union of two sets, or the area where either $A$ or $B$ is true.

\begin{tikzpicture}
	\fill[gray] \circleA;
	\fill[gray] \circleB;
	\draw \rectangleU;
	\draw \circleA node {$A$};
	\draw \circleB node {$B$};
	\draw node at (0,3.5) {\large{$A \cup B$}};
\end{tikzpicture}

\pagebreak

Second is $\cap$, the \textbf{and} operator. This is the intersection of two sets, or the area where both $A$ and $B$ are true.

\begin{tikzpicture}
	\begin{scope}
		\clip \circleA;
		\fill[gray] \circleB;	
	\end{scope}
	\draw \rectangleU;
	\draw \circleA node {$A$};
	\draw \circleB node {$B$};
	\draw node at (0,3.5) {\large{$A \cap B$}};
\end{tikzpicture}

Finally, $A^c$, also written as $A'$, the \textbf{complement} of $A$. This is also known as the \textbf{not} operator. This is the area where $A$ is false.

\begin{tikzpicture}
	\fill[gray] \rectangleU;
	\fill[white] \circleA;
	\draw \rectangleU;
	\draw \circleA node {$A$};
	\draw \circleB node {$B$};
	\draw node at (0,3.5) {\large{$A^c$}};
\end{tikzpicture}

\pagebreak

\section{Combining Venn Diagram Notation}

You can combine these operators to create more complex areas. For example, $A \cap B^c$ is the area where $A$ is true and $B$ is false.

\begin{tikzpicture}
	\fill[gray] \circleA;
	\begin{scope}
		\clip \circleA;
		\fill[white] \circleB;
	\end{scope}
	\draw \rectangleU;
	\draw \circleA node {$A$};
	\draw \circleB node {$B$};
	\draw node at (0,3.5) {\large{$A \cap B^c$}};
\end{tikzpicture}

\section{Venn Diagram Examples}

\begin{tikzpicture}
	\fill[gray] \rectangleU;
	\fill[white] \circleA;
	\fill[white] \circleB;
	\draw \rectangleU;
	\draw \circleA node {$A$};
	\draw \circleB node {$B$};
	\draw node at (0,3.5) {\large{$(A \cap B)^c$}};
\end{tikzpicture}

\begin{tikzpicture}
	\fill[gray] \rectangleUthree;
	\fill[white] \circleBthree;
	\fill[gray] \circleAthree;
	\fill[gray] \circleCthree;
	\draw \rectangleUthree;
	\draw \circleAthree node {$A$};
	\draw \circleBthree node {$B$};
	\draw \circleCthree node {$C$};
	\draw node at (0, 5.5) {\large{$((A \cup C)^c \cap B)^c$}};
\end{tikzpicture}

\begin{tikzpicture}
	\fill[gray] \circleAthree;
	\fill[gray] \circleBthree;
	\fill[white] \circleCthree;
	\draw \rectangleUthree;
	\draw \circleAthree node {$A$};
	\draw \circleBthree node {$B$};
	\draw \circleCthree node {$C$};
	\draw node at (0, 5.5) {\large{$C^c \cap (A \cup B)$}};
\end{tikzpicture}

\begin{tikzpicture}
	\fill[gray] \circleCthree;
	\fill[white] \circleAthree;
	\fill[gray] \circleBthree;
	\draw \rectangleUthree;
	\draw \circleAthree node {$A$};
	\draw \circleBthree node {$B$};
	\draw \circleCthree node {$C$};
	\draw node at (0, 5.5) {\large{$(C \cap A^c) \cup B$}};
\end{tikzpicture}

\begin{tikzpicture}
	\fill[gray] \rectangleUthree;
	\fill[white] \circleBthree;
	\fill[gray] \circleAthree;
	\fill[white] \circleCthree;
	\draw \rectangleUthree;
	\draw \circleAthree node {$A$};
	\draw \circleBthree node {$B$};
	\draw \circleCthree node {$C$};
	\draw node at (0, 5.5) {\large{$(A \cap C^c) \cup (B \cup C)^c$}};
\end{tikzpicture}

\begin{tikzpicture}
	\fill[gray] \rectangleUthree;
	\begin{scope}
		\clip \circleAthree;
		\fill[white] \circleBthree;
	\end{scope}
	\begin{scope}
		\clip \circleBthree;
		\fill[white] \circleCthree;
	\end{scope}
	\draw \rectangleUthree;
	\draw \circleAthree node {$A$};
	\draw \circleBthree node {$B$};
	\draw \circleCthree node {$C$};
	\draw node at (0, 5.5) {\large{$((A \cap B) \cup (B \cap C))^c$}};
\end{tikzpicture}

\end{document}