\documentclass{report}

\input{preamble}
\input{macros}
\input{letterfonts}

\title{\Huge{Magnet Precalculus CD}\\Complex Coordinates}
\author{\huge{Devin D. Droddy}}
\date{}

\begin{document}

\maketitle
\newpage% or \cleardoublepage
% \pdfbookmark[<level>]{<title>}{<dest>}
\pdfbookmark[section]{\contentsname}{toc}
\tableofcontents
\pagebreak

\chapter{}
\section{}
Polar coordinates on the complex plain can be represented as a complex number. Any polar point $(r,\theta)$ can be represented as $r(\cos(\theta) + \sin(\theta))$, or $r\cis(\theta)$. Complex numbers are generally written as $z=...$.

The multiplication rule says that for any two complex points, $z_1$ and $z_2$, written as $z=r(\cis(\theta))$, $z_1z_2=r_1r_2\cis(\theta_1+\theta_2)$. Conversely, $\frac{z_1}{z_2}=\frac{r_1}{r_2}\cis(\theta_1-\theta_2)$

DeMoivre's Theorem is that $z^n=(r\cis(\theta))^n=r^n\cis(n\theta)$

If a complex number $z$ satisfies the equation $z^n=w$, we say that $z$ is a complex $n^{th}$ root of $w$. We can use DeMoivre's Theorem to find these distinct $n^{th}$ roots. Let $w=r(\cos(\theta)+i\sin(\theta))$ be a complex number in polar form. If $w\neq0$, $w$ has $n$ distinct roots given by the formula below, where $k=0,1,2,3,...,n-1$.

$z_k=\sqrt[n]{r}\cis(\frac{\theta+2\pi k}{n})$
\end{document}
