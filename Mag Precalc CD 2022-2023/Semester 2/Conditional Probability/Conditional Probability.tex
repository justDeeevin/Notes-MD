\documentclass{report}

\input{preamble}
\input{macros}
\input{letterfonts}

\title{\Huge{Some Class}\\Random Examples}
\author{\huge{Devin D. Droddy}}
\date{}

\begin{document}

\maketitle
\newpage% or \cleardoublepage
% \pdfbookmark[<level>]{<title>}{<dest>}
\pdfbookmark[section]{\contentsname}{toc}
\tableofcontents
\pagebreak

\chapter{}

\section{Intro to Conditional Probability}

Conditional probability is a probability that changes when new information is given. The probability of event $A$, knowing that event $B$ has ocurred is known as the probability of $A$ \textit{given} $B$, denoted as $P(A|B)$. The formula for conditional probability is $\frac{P(A \cap A)}{P(B)}$.

\section{Two-way Tables}

\dfn{Two-way Table}{A table that records data that pertains to two different categories.}

\ex{Two-way Table}{
	\begin{tabular}{ |c|c|c| }
		\hline
		& Boys & Girls \\
		\hline 
		Prefer Football & 18 & 6 \\
		\hline
		Prefer Hockey & 10 & 16 \\
		\hline
	\end{tabular}
}

We can use two-way tables to find conditional probabilities. For instance,
\[
	P(\text{Prefers Football} | \text{Boy})=\frac{\frac{18 + 6}{18 + 6 + 10 + 16} \cdot \frac{18 + 10}{18 + 6 + 10 + 16}}{\frac{18 + 6}{18 + 6 + 10 + 16}}=\frac{\frac{24}{50} \cdot \frac{28}{50}}{\frac{24}{50}}=\frac{\frac{672}{2500}}{\frac{24}{50}}=\frac{14}{25}
\]

\end{document}