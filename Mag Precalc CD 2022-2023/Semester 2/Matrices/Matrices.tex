\documentclass{report}

\input{preamble}
\input{macros}
\input{letterfonts}

\title{\Huge{Magnet Precalculus CD}\\Matrices}
\author{\huge{Devin D. Droddy}}
\date{}

\begin{document}

\maketitle
\newpage% or \cleardoublepage
% \pdfbookmark[<level>]{<title>}{<dest>}
\pdfbookmark[section]{\contentsname}{toc}
\tableofcontents
\pagebreak

\chapter{}

\section{Introduction to Matrices}

\dfn{Matrix}{
    A \textbf{matrix} is a rectangular array of variables or constants in rows or columns, usually enclosed in brackets. These constants or variables are known as \textbf{elements}.
}

\dfn{Element}{
    An element is an individual value within a matrix. Given a matrix $A$, a given element in side of $A$ is notated as $A_{xy}$, where $x$ is the row and $y$ is the column in which the element is located.
}

\begin{note}
    If either the width or height of a matrix is more than one digit, $x$ and $y$ in element notation are generally separated by a dash (e.g. $A_{10-4}$)
\end{note}

\ex{Find an Element of a Matrix}{
    $
        A=
        \begin{bmatrix}
            -8 & 40 & 0 & -1 & 21 \\
            27 & 32 & -29 & 6 & -2 \\
            5 & -7 & 14 & 52 & -35
        \end{bmatrix}
    $

    $A_{12}=40$

    $A_{34}=52$
}

A matrix with $m$ rows and $n$ columns is known as an "$m$ by $n$" matrix, written as $m$x$n$. These are its \textbf{dimensions}.

\ex{Dimensions of a matrix}{
    Let matrix
    $
        A=
        \begin{bmatrix}
            1 & -8 \\
            -4 & 13 \\
            -6 & -2 \\
            28 & 0
        \end{bmatrix}
    $. 
    $A$ has 4 rows and 2 columns, so its dimensions are 4x2.
}

\section{Summation of Matrices}

Matrices can be summed \textbf{only if their dimensions are the same}. The process is as simple as summing all corresponding elements.

\ex{Sum of Two Matrices}{
    $
        W=
        \begin{bmatrix}
            -1 & 9 \\
            -11 & 15 \\
            8 & -20
        \end{bmatrix}
    $

    $
        Z=
        \begin{bmatrix}
            -3 & -2 \\
            -16 & 0 \\
            12 & 9
        \end{bmatrix}
    $

    $
        W+Z=
        \begin{bmatrix}
            -4 & 7 \\
            -27 & 15 \\
            20 & -11
        \end{bmatrix}
    $

    $
        W-Z=
        \begin{bmatrix}
            2 & 11 \\
            5 & 15 \\
            -4 & -29
        \end{bmatrix}
    $
}

\section{Multiplication of Matrices}

Before we get to the method of matrix multiplication, there is a very important condition that must be met.

Consider two matrices, $A$ and $B$. \textbf{They can only be multiplied if $A$ has the same number of columns as $B$ has rows}. In other words, if $A$ had dimensions $m_1$x$n_1$ and $B$ had $m_2$x$n_2$, they could only be multiplied if $n_1=m_2$. The dimensions of the product matrix are $m_1$x$n_2$

Element $AB_{hk}=A_{h1}B_{1k}+A_{h2}B_{2k}+A_{h3}B_{3k}+...+A_{hn_2}B_{n_2k}$. So, if 
$
    A=
    \begin{bmatrix}
        a & b \\
        c & d
    \end{bmatrix}
$ 
and 
$
    B=
    \begin{bmatrix}
        e & f \\
        g & h
    \end{bmatrix}
$, 
then 
$
    AB=
    \begin{bmatrix}
        ae+bg & af+bh \\
        ce+dg & cf+ch
    \end{bmatrix}
$

\ex{Multiplication of Matrices}{
    $
        A=
        \begin{bmatrix}
            9 & -5 \\
            -2 & 4
        \end{bmatrix}
    $

    $
        B=
        \begin{bmatrix}
            -3 & 13 & -5 \\
            -1 & -7 & 2
        \end{bmatrix}
    $

    $
        AB=
        \begin{bmatrix}
            (9 \cdot -3)+(-5 \cdot -1) & (9 \cdot 13)+(-5 \cdot -7) & (9 \cdot -5)+(-5 \cdot 2) \\
            (-2 \cdot -3)+(4 \cdot -1) & (-2 \cdot 13)+(4 \cdot -7) & (-2 \cdot -5)+(4 \cdot 2)
        \end{bmatrix}
        =
        \begin{bmatrix}
            -27+5 & 117+35 & -45-10 \\
            6-4 & -26-28 & -10+8
        \end{bmatrix}
        =
        \begin{bmatrix}
            -22 & 152 & -55 \\
            2 & -54 & -2
        \end{bmatrix}
    $
}

\section{Determinant of a Matrix}

Every square matrix has a real number that is its \textbf{determinant}. The determinant of matrix $A$ is denoted as $\det(A)$ or $|A|$.

The determinant of a 2x2 matrix is called a \textbf{second-order determinant}. To find a second-order determinant, use the following formula: $|A|=A_{11}A_{22}-A_{21}A_{12}$

\ex{Determinant of a 2x2 Matrix} {
    $
        A=
        \begin{bmatrix}
            -4 & 3 \\
            5 & -10
        \end{bmatrix}
    $

    $|A|=(-4 \cdot -10)-(5 \cdot 3)=40-15=25$
}

The determinant of a 3x3 matrix is called a \textbf{third-order} determinant. To find a third-order determinant, use the steps below:

\begin{enumerate}
    \item Rewrite the first two columns to the right of the matrix
    \item Find the sum of the products of each downward diagonal
    \item Find the sum of the products of each upward diagonal
    \item Subtract the upward diagonal sum from the downward diagonal sum
\end{enumerate}

\ex{Determinant of a 3x3 Matrix} {
    $
        A=
        \begin{bmatrix}
            3 & -7 & 2 \\
            5 & 4 & -5 \\
            1 & 5 & -1
        \end{bmatrix}
    $

    $   
        \begin{bmatrix}
            3 & -7 & 2 \\
            5 & 4 & -5 \\
            1 & 5 & -1
        \end{bmatrix}
        \begin{matrix}
            3 & -7 \\
            5 & 4 \\
            1 & 5
        \end{matrix}
    $ \\

    Downward diagonal one: $[3,4,-1]$ \\
    Downward diagonal two: $[-7,-5,1]$ \\
    Downward diagonal three: $[2,5,5]$ \\
    
    Upward diagonal one: $[1,4,2]$ \\
    Upward diagonal two: $[5,-5,3]$ \\
    Upward diagonal three: $[-1,5,-7]$ \\

    $(3 \cdot 4 \cdot -1)+(-7 \cdot 5 \cdot 1)+(2 \cdot 5 \cdot 5)=-12-35+50=3$

    $(1 \cdot 4 \cdot 2)+(5 \cdot -5 \cdot 3)+(-1 \cdot 5 \cdot -7)=8-75+35=-32$

    $3+32=\boxed{35}$
}

\end{document}
