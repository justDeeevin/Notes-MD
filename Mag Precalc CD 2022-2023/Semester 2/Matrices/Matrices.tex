\documentclass{report}

\input{preamble}
\input{macros}
\input{letterfonts}

\title{\Huge{Magnet Precalculus CD}\\Matrices}
\author{\huge{Devin D. Droddy}}
\date{}

\begin{document}

\maketitle
\newpage% or \cleardoublepage
% \pdfbookmark[<level>]{<title>}{<dest>}
\pdfbookmark[section]{\contentsname}{toc}
\tableofcontents
\pagebreak

\chapter{}

\section{Introduction to Matrices}

\dfn{Matrix}{
    A \textbf{matrix} is a rectangular array of variables or constants in rows or columns, usually enclosed in brackets. These constants or variables are known as \textbf{elements}.
}

\dfn{Element}{
    An element is an individual value within a matrix. Given a matrix $A$, a given element in side of $A$ is notated as $A_{xy}$, where $x$ is the row and $y$ is the column in which the element is located.
}

\begin{note}
    If either the width or height of a matrix is more than one digit, $x$ and $y$ in element notation are generally separated by a dash (e.g. $A_{10-4}$)
\end{note}

\ex{Find an Element of a Matrix}{
    $
        A=
        \begin{bmatrix}
            -8 & 40 & 0 & -1 & 21 \\
            27 & 32 & -29 & 6 & -2 \\
            5 & -7 & 14 & 52 & -35
        \end{bmatrix}
    $

    $A_{12}=40$

    $A_{34}=52$
}

\section{Summation of Matrices}

Matrices can be summed \textbf{only if their dimensions are the same}. The process is as simple as summing all corresponding elements.

\ex{Sum of Two Matrices}{
    $
        W=
        \begin{bmatrix}
            -1 & 9 \\
            -11 & 15 \\
            8 & -20
        \end{bmatrix}
    $

    $
        Z=
        \begin{bmatrix}
            -3 & -2 \\
            -16 & 0 \\
            12 & 9
        \end{bmatrix}
    $

    $
        W+Z=
        \begin{bmatrix}
            -4 & 7 \\
            -27 & 15 \\
            20 & -11
        \end{bmatrix}
    $

    $
        W-Z=
        \begin{bmatrix}
            2 & 11 \\
            5 & 15 \\
            -4 & -29
        \end{bmatrix}
    $
}

\end{document}
