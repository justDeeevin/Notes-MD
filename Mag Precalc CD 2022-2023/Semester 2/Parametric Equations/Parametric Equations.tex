\documentclass{report}

\input{preamble}
\input{macros}
\input{letterfonts}

\title{\Huge{Magnet Precalculus CD}Parametric Equations}
\author{\huge{Devin D. Droddy}}
\date{}

\begin{document}

\maketitle
\newpage% or \cleardoublepage
% \pdfbookmark[<level>]{<title>}{<dest>}
\pdfbookmark[section]{\contentsname}{toc}
\tableofcontents
\pagebreak

\chapter{}
\section{Intro to Parametric Equations}

We can represent the combination of multiple functions on one 2d plane by defining both $x$ and $y$ in terms of a parameter, often $t$. These two equations are known as parametric equations. When graphing parametric equations, you should draw arrows between the points you plot, in the direction that $t$ is moving.

% \ex{Plotting a Parametric Equation}{
%     $x=t^2;y=3t-2;-2\le t\le 4$
%     \begin{tabular}{|c|c|c|}
%         \hline 
%         t & x & y 
%         \hline 
%         -2 & 4 & -8 
%         \hline 
%         -1 & 1 & -5 
%         \hline 
%         0 & 0 & -2
%         \hline 
%         1 & 1 & 1 
%         \hline 
%         2 & 4 & 4
%         \hline 
%         3 & 9 & 7 
%         \hline 
%         4 & 16 & 10 
%         \hline 
%     \end{tabular}
%     \begin{center}
%         % \begin{tikzpicture}
%            
%         % \end{tikzpicture}
%     \end{center}
% }

\section{Parametric Equations in Rectangular Form}

To put a parametric equation in rectangular form, solve one of the equations for t, substitute the resulting expression into the other equation, and simplify.

\qs{Write the following pair of parametric equations in rectangular form.}{
    $x=4t-1$

    $y=6-t$
}

$t=6-y$

$x=4(6-y)-1=24-4y-1=23-4y$

$x-23=-4y$

$\boxed{y=\frac{-x+23}{4}}$

\qs{Write the following pair of parametric equations in rectangular form.}{
    $x=\frac{t+2}{t}$

    $y=\frac{1}{t}$
}

$t=\frac{1}{y}$

$x=\frac{\frac{1}{y}+2}{\frac{1}{y}}=\frac{\frac{1}{y}}{\frac{1}{y}}+\frac{2}{\frac{1}{y}}=1+\frac{2}{\frac{1}{y}}=1+2y$

$x-1=2y$

$\boxed{y=\frac{x-1}{2}}$


\section{Polar \& Parametric Equations}

The graph of a polar equation $r=f(\theta)$ is the same as the graph of the parametric equations $x=f(\theta)\cos(\theta)$ and $y=f(\theta)\sin(\theta)$.

\qs{Write the following pair of parametric equations in rectangular form}{
    $x=3\cos(\theta)$

    $y=2\sin(\theta)$
}

$\cos(\theta)=\frac{x}{3}$

$\cos^2(\theta)=\frac{x^2}{9}$

$\sin(\theta)=\frac{y}{2}$

$\sin^2(\theta)=\frac{y^2}{4}$

$\sin^2(\theta)+\cos^2(\theta)=1$

$\boxed{\frac{x^2}{9}+\frac{y^2}{4}=1}$

\qs{Write the following pair of parametric equations in rectangular form}{
    $x=\sin^2(\theta)$

    $y=4\cos(\theta)$
}

$\cos(\theta)=\frac{y}{4}$

$\cos^2(\theta)=\frac{y^2}{16}$

$\boxed{x+\frac{y^2}{16}=1}$

\end{document}