\documentclass{report}

\input{preamble}
\input{macros}
\input{letterfonts}

\title{\Huge{Magnet Precalculus CD}\\Polar coordinates}
\author{\huge{Devin D. Droddy}}
\date{}

\begin{document}
\usepgfplotslibrary{polar}
    \maketitle
    \newpage% or \cleardoublepage
    % \pdfbookmark[<level>]{<title>}{<dest>}
    \pdfbookmark[section]{\contentsname}{toc}
    \tableofcontents
    \pagebreak

    \chapter{Introduction to Polar Coordinates}

    Normally, points and functions are represented rectangularly, with an $x$ and a $y$ coordinate. However, they can also be represented in polar form.
    \dfn{Polar Coordinates}{
        A coordinate represented by a distance from the origin $r$ and an angle from the positive x-axis $\theta$. Polar coordinates are in the form $(r,\theta)$.
    }

    In polar functions, the dependent variable is generally $r$. With polar, you can far more easily graph circles and other curves. For instance, take $r=3$:
    \begin{center}
        \begin{tikzpicture}
            \begin{polaraxis}[
                axis lines = box,
            ]
            \addplot [
                red,
                domain=0:360,
                samples=200,
                smooth,
                data cs=polar,
            ](x,{3});
            \end{polaraxis}
        \end{tikzpicture}
    \end{center}

    \chapter{Polar function types}
    \section{Circle}

    Circles in polar form are fairly simple. There are three kinds:
    \begin{itemize}
        \item $r=n$
        \item $r=n\sin(\theta)$
        \item $r=n\cos(\theta)$
    \end{itemize}

    In the case of $r=n$, the circle is centered on the pole (or the origin), and has a radius of $|n|$.
    \ex{$r=n$ circle}{
        \begin{center}
            \begin{tikzpicture}
                \begin{polaraxis}[
                    axis lines = box,
                ]
                    \addplot [
                        red,
                        domain=0:360,
                        samples=200,
                        smooth,
                        data cs=polar,
                    ](x,{3});
                    \addlegendentry{$r=3$}

                    \addplot [
                        blue,
                        domain=0:360,
                        samples=200,
                        smooth,
                        data cs=polar,
                    ](x,{1});
                    \addlegendentry{$r=1$}
                \end{polaraxis}
            \end{tikzpicture}
        \end{center}
    }

    In the case of $r=n\cos(\theta)$, the circle is centered at (polar) $(\frac{n}{2},0)$ and has a radius of $\frac{n}{2}$. You can also think of it as being centered on the polar axis (or positive x-axis), tangent to the pole, and tangent to the polar point $(n,0)$.
    \ex{$r=n\cos(\theta)$ circle}{
        \begin{center}
            \begin{tikzpicture}
                \begin{polaraxis}[
                    axis lines = box
                ]
                    \addplot[
                        red,
                        domain=0:360,
                        samples=200,
                        smooth,
                        data cs=polar,
                    ](x, {2*cos(x)});
                    \addlegendentry{$r=2\cos(\theta)$};
                    \addplot[
                        blue,
                        domain=0:360,
                        samples=200,
                        smooth,
                        data cs=polar,
                    ](x, {-5*cos(x)});
                    \addlegendentry{$r=-5\cos(\theta)$}
                \end{polaraxis}
            \end{tikzpicture}
        \end{center}
    }

    The function $r=n\sin(\theta)$ is the same as $r=n\cos(\theta)$, but instead of being centered on the polar axis, it is centered on the polar line $\theta=\frac{\pi}{2}$ (or the y-axis).

    \ex{$r=n\sin(\theta)$ circle}{
        \begin{center}
            \begin{tikzpicture}
                \begin{polaraxis}[
                    axis lines = box
                ]
                    \addplot[
                        red,
                        domain=0:360,
                        samples=200,
                        smooth,
                        data cs=polar,
                    ](x, {-1*sin(x)});
                    \addlegendentry{$r=-1\sin(\theta)$};

                    \addplot[
                        blue,
                        domain=0:360,
                        samples=200,
                        smooth,
                        data cs=polar,
                    ](x, {5*sin(x)});
                    \addlegendentry{$r=5\sin(\theta)$};
                \end{polaraxis}
            \end{tikzpicture}
        \end{center}
    }
\end{document}