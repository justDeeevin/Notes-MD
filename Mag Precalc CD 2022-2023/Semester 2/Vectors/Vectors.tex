\documentclass{report}

\input{preamble}
\input{macros}
\input{letterfonts}

\title{\Huge{Magnet Precalculus CD}\\Vectors}
\author{\huge{Devin D. Droddy}}
\date{}

\begin{document}

\maketitle
\newpage% or \cleardoublepage
% \pdfbookmark[<level>]{<title>}{<dest>}
\pdfbookmark[section]{\contentsname}{toc}
\tableofcontents
\pagebreak

\chapter{Geometric Representation of Vectors}

\section{Combination }

\section{Magnitude of Vectors}

The magnitude (length) of a vector $v=\left\langle a,b \right\rangle$ is $||v||=\sqrt{a^2+b^2}$. If vector $v$ is represented by the arrow from $(x_1,y_1)$ to $(x_2,y_2)$, then $||v||=\sqrt{(x_2-x_1)^2+(y_2-y_1)^2}$.

\section{Unit Vector}

\dfn{Unit Vector}{
	A vector $u$ for which $||u||=1$ is called a \textbf{unit vector}.

	\begin{center}
		$u=\frac{v}{||v||}$

		$i=\left\langle 1,0 \right\rangle$ and $j=\left\langle 0,1 \right\rangle$
	\end{center}
}

\section{Components of a Vector}

Let $v$ be a vector with magnitude $||v||$ and direction $\theta$. Then, $v=\left\langle a,b \right\rangle=ai+bj$ where $a=||v||\cos(\theta)$ and $b=||v||\sin(\theta)$. Therefore, we can express $v$ as $v=||v||\cos(\theta)i+||v||\sin(\theta)j$

\chapter{The Language of Vectors}

\section{The Dot Product}

\dfn{Dot Product}{
	A \textbf{\ul{dot product}} of $u=\left\langle a_1,b_1 \right\rangle$ and $v=\left\langle a_2,b_2 \right\rangle$ is $u \bullet v=a_1a_2+b_1b_2$
}

Another form of the dot product is $u \bullet v=||u||||v||\cos(\theta)$, where $\theta$ is the angle between $u$ and $v$.\\

To find the angle between two vectors, the dot product can be used:

Let $\theta$ be the angle between two vectors, $u$ and $v$. Where $0\le\theta\le\pi$, $\cos(\theta)=\frac{u \bullet v}{||u||||v||}$\\

The dot product can be used to find the magnitude of a vector. $||u||=u \bullet u=u^2$

\dfn{Orthogonal Vectors}{
	When two angles are orthogonal (a.k.a. perpendicular or normal), their dot product is 0.
}

\qs{}{
	Find the angle between the vectors $u=2i-j$ and $v=6i+4j$
}

\sol $\cos(\theta)=\frac{(2 \bullet 6)+(-1 \bullet 4)}{\sqrt{2^2+(-1)^2}\bullet\sqrt{6^2+4^2}}=\frac{12-4}{\sqrt{5}\bullet\sqrt{52}}=\frac{8}{\sqrt{260}}=\frac{8}{2\sqrt{65}};\theta=\cos^{-1}(\frac{8}{1\sqrt{65}})\approx \boxed{60.255^\circ}$

\end{document}