\documentclass{report}

\input{preamble}
\input{macros}
\input{letterfonts}

\title{\Huge{Magnet Precalculus CD}\\Vectors}
\author{\huge{Devin D. Droddy}}
\date{}

\begin{document}

\maketitle
\newpage% or \cleardoublepage
% \pdfbookmark[<level>]{<title>}{<dest>}
\pdfbookmark[section]{\contentsname}{toc}
\tableofcontents
\pagebreak

\chapter{Geometric Representation of Vectors}

\section{Combination }

\section{Magnitude of Vectors}

The magnitude (length) of a vector $v=\left\langle a,b \right\rangle$ is $||v||=\sqrt{a^2+b^2}$. If vector $v$ is represented by the arrow from $(x_1,y_1)$ to $(x_2,y_2)$, then $||v||=\sqrt{(x_2-x_1)^2+(y_2-y_1)^2}$.

\section{Unit Vector}

\dfn{Unit Vector}{
	A vector $u$ for which $||u||=1$ is called a \textbf{unit vector}.

	\begin{center}
		$u=\frac{v}{||v||}$

		$i=\left\langle 1,0 \right\rangle$ and $j=\left\langle 0,1 \right\rangle$
	\end{center}
}

\section{Components of a Vector}

Let $v$ be a vector with magnitude $||v||$ and direction $\theta$. Then, $v=\left\langle a,b \right\rangle=ai+bj$ where $a=||v||\cos(\theta)$ and $b=||v||\sin(\theta)$. Therefore, we can express $v$ as $v=||v||\cos(\theta)i+||v||\sin(\theta)j$

\end{document}
